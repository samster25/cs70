\documentclass[11pt,letterpaper]{article}
\usepackage{amsmath}
\usepackage{amssymb}
\usepackage{fullpage}
\usepackage{enumerate}

\title{CS 70, Summer 2014 --- Homework 1} % Put the correct homework number here.
\author{Harsimran (Sammy) Sidhu, SID 23796591} % Put your name and student ID here.

\begin{document}

\maketitle

Collaboraters: Chonyi Lama, Jenny Pushkarskaya

Sources: http://comet.lehman.cuny.edu/sormani/teaching/induction.html

\section*{Problem 1} % Put the correct problem number here

\begin{enumerate}[(a)]

\item
\begin{table}[h] 
\caption{$P\wedge (Q\vee P) \equiv P \wedge Q$} % title of Table 
\centering % used for centering table 
\begin{tabular}{c c c c c} % centered columns (4 columns) 
\hline %inserts double horizontal lines 
$P$ & $Q$ & $Q\vee{P}$ & $P\wedge{(Q\vee P)}$ & $P\wedge Q$ \\ % inserts table 
%heading 
\hline % inserts single horizontal line 
T & T & T & T & T \\
T & F & T & T & F \\
F & T & T & F & F \\
F & F & F & F & F \\ % [1ex] adds vertical space 
\hline %inserts single line 
\end{tabular} 
\label{table:1a} % is used to refer this table in the text 
\end{table}

$Not~Equivalent$


\item 

\begin{table}[h] 
\caption{$(P\Rightarrow Q)\Rightarrow R \equiv P\Rightarrow (Q\Rightarrow R)$} % title of Table 
\centering % used for centering table 
\begin{tabular}{c c c c c c c} % centered columns 
\hline %inserts double horizontal lines 
$P$ & $Q$ & $R$ \vline & $(P\Rightarrow Q)$ \vline & $(P\Rightarrow Q)\Rightarrow R$ \vline & $P\Rightarrow (Q\Rightarrow R)$ \vline & $(Q\Rightarrow R)$ \\ % inserts table 
%heading 
\hline % inserts single horizontal line 
T & T & T & T & T & T & T\\
T & T & F & T & F & F & F\\
T & F & T & F & T & T & T\\
T & F & F & F & T & T & T\\
F & F & F & T & F & T & T\\
F & F & T & T & T & T & T\\
F & T & F & T & F & T & F\\
F & T & T & T & T & T & T\\
\hline %inserts single line 
\end{tabular} 
\label{table:1a} % is used to refer this table in the text 
\end{table}

$Not~Equivalent$
\clearpage
\item 
\begin{table}[h] 
\caption{$(P\Rightarrow Q)\Rightarrow (P\Rightarrow R) \equiv P\Rightarrow (Q\Rightarrow R)$} % title of Table 
\centering % used for centering table 
\begin{tabular}{c c c c c c c c} % centered columns 
\hline %inserts double horizontal lines 
$P$ & $Q$ & $R$ \vline & $(P\Rightarrow Q)$ \vline & $(P\Rightarrow R)$ \vline & $(P\Rightarrow Q)\Rightarrow (P\Rightarrow R)$ \vline & $(Q\Rightarrow R)$ \vline & $P\Rightarrow (Q\Rightarrow R)$\\ % inserts table 
%heading 
\hline % inserts single horizontal line 
T & T & T & T & T & T & T & T\\
T & T & F & T & F & F & F & F\\
T & F & T & F & T & T & T & T\\
T & F & F & F & F & T & T & T\\
F & F & F & T & T & T & T & T\\
F & F & T & T & T & T & T & T\\
F & T & F & T & T & T & F & T\\
F & T & T & T & T & T & T & T\\
\hline %inserts single line 
\end{tabular} 
\label{table:1a} % is used to refer this table in the text 
\end{table}
$Equivalent$

\item

\begin{table}[h] 
\caption{$(P\wedge\neg Q)\Leftrightarrow (\neg P\vee Q)\equiv (Q\wedge \neg P)\Leftrightarrow (\neg Q \vee P)$} % title of Table 
\centering % used for centering table 
\begin{tabular}{c c c c c c c c} % centered columns 
\hline %inserts double horizontal lines 
$P$ & $Q$ \vline & $(P\wedge\neg Q)$ \vline & $(\neg P\vee Q)$ \vline & $(P\wedge\neg Q)\Leftrightarrow (\neg P\vee Q)$ \vline & $(Q\wedge \neg P)$ \vline & $(\neg Q \vee P)$ \vline & $(Q\wedge \neg P)\Leftrightarrow (\neg Q \vee P)$\\ % inserts table 
%heading 
\hline % inserts single horizontal line 
T & T & F & T & F & F & T & F \\
T & F & T & F & F & F & T & F \\
F & T & F & T & F & T & F & F \\
F & F & F & T & F & F & T & F \\

\hline %inserts single line 
\end{tabular} 
\label{table:1a} % is used to refer this table in the text 
\end{table}
$Equivalent$

\end{enumerate}

\newpage

\section*{Problem 2}

\begin{enumerate}[(a)]
%2a
\item
\begin{enumerate}[(I)]
\item
$\forall x (P(x)\Rightarrow B(x))$
\item
$\forall x (U(x)\Rightarrow \neg F(x))$
\item
$\forall x (O(x)\Rightarrow N(x))$
\item
$\forall x (B(x)\Rightarrow F(x))$
\item
$\forall x (K(x)\Rightarrow P(x))$
\item
$\forall x (N(x)\Rightarrow U(x))$
\end{enumerate}

%2b
\item
\begin{enumerate}[(I)]
\item
$\forall x (\neg B(x)\Rightarrow \neg P(x))$
\item
$\forall x (F(x)\Rightarrow \neg U(x))$
\item
$\forall x (\neg N(x)\Rightarrow \neg O(x))$
\item
$\forall x (\neg F(x)\Rightarrow \neg B(x))$
\item
$\forall x (\neg P(x)\Rightarrow \neg K(x))$
\item
$\forall x (\neg U(x)\Rightarrow \neg N(x))$
\end{enumerate}

%2c
\item
If a person wears kid gloves, they go to a party, brush their hair, look fascinating, are tidy, have self control, and aren't opium-eaters.

$\forall x (K(x)\Rightarrow P(x)\Rightarrow B(x)\Rightarrow F(x)\Rightarrow \neg U(x)\Rightarrow \neg N(x)\Rightarrow \neg O(x))$


\end{enumerate}

\clearpage
\section*{Problem 3}

\begin{enumerate}[(a)]
%3a
\item $\forall x \exists y ~(xy\geq x^2)$ True\\
$Case~0: x = 0 \rightarrow 0\geq 0$\\
$Case~1: \forall x\in \mathbb{R}^{+}~\exists y
~((y\geq x)\Rightarrow (xy\geq x^2))$\\
$Case~2: \forall x\in \mathbb{R}^{-}~\exists y
~((y\leq x)\Rightarrow (xy\geq x^2))$
%3b
\item $\exists y\forall x ~(xy\geq x^2)$ False\\
$\forall x\in\mathbb{R}~(x^2\geq 0)\\
Case~0: y=0,\forall x\neq0~(0 < x^2)\Rightarrow\neg(xy\geq x^2)~ \\
Case~1: \exists y\in\mathbb{R}^+~\forall x\
((x<0)\Rightarrow (xy < 0)\Rightarrow\neg(xy\geq x^2))\\
Case~2: \exists y\in\mathbb{R}^-~\forall x\
((x>0)\Rightarrow (xy <0)\Rightarrow\neg(xy\geq x^2))\\$
%3c
\item $\neg\forall x\exists y~ (xy>0\Rightarrow y>0)$ False\\
$\neg\forall x\exists y~(\neg(xy>0)\vee(y>0))\\
\exists x\forall y~ \neg(\neg(xy>0)\vee(y>0))\\
\exists x\forall y~ ((xy>0)\wedge\neg(y>0))\\
\exists x\forall y~ ((xy>0)\wedge(y\leq0))\\$
\smallskip

False whenever y is greater than 0 $(y>0)$ which disagrees with $\forall y$

\end{enumerate}
\clearpage

\section*{Problem 4}

\begin{enumerate}[(a)]
%4a
\item
$\neg\forall x~\exists y~(P(x)\Rightarrow\neg Q(x,y)) \equiv \exists x~\forall y~(P(x)\wedge Q(x,y))$ \hfill Original Statement\\
$\exists x~\forall y~\neg(P(x)\Rightarrow\neg Q(x,y)) \equiv$\hfill factor in negation\\
$\exists x~\forall y~\neg(\neg P(x)\vee\neg Q(x,y)) \equiv$ \hfill Implication to Or\\
$\exists x~\forall y(P(x)\wedge Q(x,y)) \equiv$ \hfill Demorgan's Law\\ 
\\
Equivalent
%4b
\item
$\forall x~\exists y~(P(x)\Rightarrow Q(x,y)) \equiv \forall x~(P(x)\Rightarrow (\exists y~ Q(x,y)))$\hfill Original Statement\\
$\forall x~\exists y~(\neg P(x)\vee Q(x,y)) \equiv$\hfill Implication to Or\\
$\forall x~((\exists y~\neg P(x))\vee (\exists y~Q(x,y))) \equiv$\hfill Distribution of quantifier\\
$\forall x~(\neg(\forall y~ P(x))\vee (\exists y~Q(x,y))) \equiv$\hfill Factor out negation\\
$\forall x~((\forall y~ P(x))\Rightarrow (\exists y~Q(x,y))) \equiv$\hfill Or to Implication\\
$\forall x~(P(x)\Rightarrow (\exists y~Q(x,y))) \equiv$ \hfill Obvious\\
\\
Equivalent
%4c
\item
$\forall x~\exists y~(Q(x,y)\Rightarrow P(x)) \equiv \forall x~(\exists y~ Q(x,y)\Rightarrow P(x))$\hfill Original Statement\\
$\forall x~\exists y~(\neg Q(x,y)\vee P(x)) \equiv$\hfill Implication to Or\\
$\forall x~((\exists y~\neg Q(x,y))\vee (\exists y~ P(x))) \equiv$\hfill Distribution of quantifier\\
$\forall x~(\neg(\forall y~ Q(x,y))\vee (\exists y~ P(x))) \equiv$\hfill Factor out negation\\
$\forall x~((\forall y~ Q(x,y))\Rightarrow (\exists y~ P(x))) \equiv$\hfill Or to Implication\\
$\forall x~((\forall y~ Q(x,y))\Rightarrow P(x)) \equiv$\hfill Obvious\\
$\forall x~((\forall y~ Q(x,y))\Rightarrow P(x)) \not\equiv \forall x~(\exists y~ Q(x,y)\Rightarrow P(x))$\hfill Invalid\\
\\
Not Equivalent
\end{enumerate}

\clearpage

\section*{Problem 5}
\begin{enumerate}[(a)]
%5a
\item
$\forall n\in\mathbb{N}~(n~odd \Rightarrow n^2 + 2n~odd)$\hfill Original Statement\\
$Assume~n~is~odd\\
\forall n\in\mathbb{N}~\exists k\in\mathbb{Z}~(n~odd\Rightarrow n=2k+1)$\hfill Definition of an odd number\\
$\forall n\in\mathbb{N}~\exists d\in\mathbb{Z}~(n^2+2n = 2d+1 \Rightarrow n^2+2n~odd)\hfill n^2+2n$ is odd if d exists\\
Substitution\\
$(2k+1)^2 + 2(2k+1) = (4k^2 + 4k +1)+4k +2 = (4k^2 +8k + 2) + 1 = 2(2k^2 + 4k +1) + 1$\\
$d=2k^2 +4k +1\hfill d$ Exists\\
\\
True, direct proof.
\\
%5b
\item
$\forall n\in\mathbb{N}~(n^2 +7n +1~odd)$\hfill Original Statement\\
Case 1: $n~is~odd$\\
$\forall n\in\mathbb{N}~\exists k\in\mathbb{Z}~(n~odd\Rightarrow n =2k+1)$ \hfill Definition of an odd number\\
$\forall n\in\mathbb{N}~\exists d\in\mathbb{Z}~(n^2+7n+1 = 2d+1 \Rightarrow n^2+7n+1~odd)\hfill n^2+7n+1$ is odd if d exists\\
Substitution\\
$(2k+1)^2 + 7(2k+1) + 1 = (4k^2 + 18k +8) + 1 = 2(2k^2 + 9k + 4) + 1$\\
$d = (2k^2 + 9k + 4)\hfill d$ Exists\\
\\
$Therefore~\forall n\in\mathbb{N}~(n~odd\Rightarrow n^2 +7n +1~odd)\hfill (n^2 +7n +1)$ is odd whenever n is odd\\

Case 2: $n~is~even$\\
$\forall n\in\mathbb{N}~\exists k\in\mathbb{Z}~(n~even\Rightarrow n =2k)$ \hfill Definition of an even number\\
$\forall n\in\mathbb{N}~\exists d\in\mathbb{Z}~(n^2+7n+1 = 2d+1 \Rightarrow n^2+7n+1~odd)\hfill n^2+7n+1$ is odd if d exists\\
Substitution\\
$(2k)^2 + 7(2k) + 1 = (4k^2 + 14k) + 1 = 2(2k^2 + 7k) + 1$\\
$d = (2k^2 + 7k)\hfill d$ Exists\\
\\
$Therefore~\forall n\in\mathbb{N}~(n~even\Rightarrow n^2 +7n +1~odd)\hfill (n^2 +7n +1)$ is odd whenever n is even\\
\\
True, Proof by Cases
\\
%5c
\item
$\forall ~a,b\in\mathbb{R}~(a+b\leq 10\Rightarrow (a\leq 7 \vee b\leq 3))$\hfill Original Statement\\
$\forall ~a,b\in\mathbb{R}~(\neg(a\leq 7 \vee b\leq 3) \Rightarrow \neg(a+b\leq 10))$\hfill Contrapositive\\
$\forall ~a,b\in\mathbb{R}~((\neg(a\leq 7) \wedge \neg(b\leq 3)) \Rightarrow \neg(a+b\leq 10))$\hfill Demorgan's Law\\
$\forall ~a,b\in\mathbb{R}~(((a> 7) \wedge (b> 3)) \Rightarrow (a+b> 10))$\hfill Applied Negation\\
\\
True, Proof by Contrapositive
\clearpage

%5d
\item
$\forall r\in\mathbb{R}~(r~irrational\Rightarrow r+1~irrational)$\hfill Original Statement\\
$Assume~\neg(r+1~irrational)\equiv (r+1~rational)$\\
$\forall r\in\mathbb{R}~\exists~a,b\in\mathbb{Z}~(r+1~rational\Leftrightarrow r+1=\frac{a}{b})(b\neq 0)$\hfill definition of a rational number\\
$r= \frac{a}{b}-1=\frac{a}{b}-\frac{b}{b}=\frac{a-b}{b}$\hfill Solving for $r$ which gives a rational number\\
\\
Assuming $\neg(r+1~rational)~r$ is rational which is a proof by contrapositive\\
\\
True, Proof by Contrapositive
\\
\item
$\forall n\in\mathbb{N}~(10n^2>n!)$\hfill Original Statement\\
Case 1: $n=6$\\
$10(6)^2 > 6!$\\
$360 \not> 720$\hfill Invalid\\

False, Proof by Counterexample
\end{enumerate}

\clearpage
\section*{Problem 6}
\begin{enumerate}[(a)]
%6a
\item
$\forall n\geq 1~\sum\limits_{i=1}^n\frac{1}{i(i+1)} = \dfrac{n}{n+1}$\hfill Original Statement\\
$Base~case:n=1\rightarrow\frac{1}{1(1+1)}=\frac{1}{1+1} = \frac{1}{2}$\hfill True\\
$Inductive~ Hypothesis: \forall k\geq 1~\sum\limits_{i=1}^k\frac{1}{i(i+1)} = \frac{k}{k+1}$\\
$Let~ n=k+1\\
\\
\sum\limits_{i=1}^{k+1}\frac{1}{i(i+1)} = \frac{k+1}{(k+1)+1}$\hfill Substitute\\
\\
$\sum\limits_{i=1}^{k+1}\frac{1}{i(i+1)} = \frac{(k+1)}{(k+2)}$\hfill Simplify\\
\\
$\sum\limits_{i=1}^{k}\frac{1}{i(i+1)} + \frac{1}{(k+1)(k+2)}= $\hfill Expand series\\
\\
$\frac{k}{(k+1)} + \frac{1}{(k+1)(k+2)}= $\hfill Substitute inductive hypothesis\\
\\
$\frac{(k^2+2k)}{(k+1)(k+2)} + \frac{1}{(k+1)(k+2)}= $\hfill Cross multiply\\
\\
$\frac{(k^2+2k + 1)}{(k+1)(k+2)} = \frac{(k+1)^2}{(k+1)(k+2)}=$\hfill Simplify and factor\\
\\
$\frac{(k+1)}{(k+2)} =\frac{(k+1)}{(k+2)}$\hfill Valid\\
\\
%6b
\item
$\forall n\in\mathbb{N}~(5|(8^n - 3^n))$\hfill Original Statement\\
$Proof:~(5|(8^n - 3^n))$\\
$Base~case:~(n=0)\rightarrow(5|(8^0-5^0))= (5|0)$\hfill True\\
$Inductive~hypothesis:~\forall k\in\mathbb{N}~(5|(8^k - 3^k))$\\
$let ~n =k+1$\\
$(5|(8^{k+1} - 3^{k+1}))= (5|(8\times 8^{k} - 3\times 3^{k}))$\hfill Expand\\
$(5|(8\times 8^{k} - 3\times 3^{k}))= (5|((8\times 8^{k}) - (8\times 3^{k}) +(8\times 3^{k})- (3\times 3^{k})))$\\
$(5|(8\times 8^{k} - 3\times 3^{k}))= (5|(8\times (8^k -3^{k}) +5\times 3^{k}))$\hfill Group factors\\
Our hypothesis states that 5 divides $(8^k - 3^k)$ so $\exists d\in\mathbb{Z}~((8^k-3^k)=5d)$\\
$(5|(8\times 8^{k} - 3\times 3^{k}))= \exists d\in\mathbb{Z}~(5|(8\times 5d +5\times 3^{k}))$\\
$(5|(8\times 8^{k} - 3\times 3^{k}))= \exists d\in\mathbb{Z}~(5|5(8d + 3^{k}))$\hfill Valid\\

\end{enumerate}
\clearpage

\section*{Problem 7}

\clearpage
\section*{Problem 8}
Postponed!!

%\begin{enumerate}[(a)]
%%8a
%\item
%
%$\forall c\geq 12~\exists(m,n)\in\mathbb{N}~(3m + 7n = c)~(c\geq 12)$\hfill Original Statement\\
%$m$ is the number of 3 cent coins and $n$ is the number of 7 cent coins\\
%$Base~Cases:$\\
%$(c=12)\rightarrow (m=4,n=0)\rightarrow (3(4) + 7(0) = 12)$\\
%$(c=13)\rightarrow (m=2,n=1)\rightarrow (3(2) + 7(1) = 13)$\hfill (m=m-2,~n=n+1)\\
%$(c=14)\rightarrow (m=0,n=2)\rightarrow (3(0) + 7(2) = 14)$\hfill (m=m-2,~n=n+1)\\
%$(c=15)\rightarrow (m=5,n=0)\rightarrow (3(5) + 7(0) = 15)$\hfill (m=m+5,~n=n-2)\\
%$Inductive~hypothesis:~\forall c\geq 12~\exists(m,n)\in\mathbb{N}~(3m + 7n = c)$\\
%$Inductive~Cases:$\\
%$Case~1:$\\
%If there is atleast two 3 cent coins $(m\geq2)$ then we can remove the two 3 cent coins and add a 7 cent coin to increment the cent count by 1\\ 
%\\
%$Case~2:$
%If there is atleast two 7 cent coins $(n\geq2)$ then we can remove the two 7 cent coins and add five 3 cent coins to increment the cent count by 1\\ 
%\\
%When $(c\geq 12)$ there is always at least two coins of m or n. That means we can always increment the cent count via the 2 cases.
%\\
%%8b
%\item
%$\forall n\in\mathbb{N}~(F_n = \frac{\phi^n-(1-\phi)^n}{\sqrt{5}})$ where $\phi = \frac{1+\sqrt{5}}{2}$\\
%$(F_n = F_{n-2} +F_{n-1})$ where $F_0 = 0,~F_1 =1$\\
%\\
%$Proof: F_n = \frac{\phi^n-(1-\phi)^n}{\sqrt{5}}= F_{n-2} +F_{n-1}$\\
%\\
%$Base~Cases:$\\
%$n=0\rightarrow \frac{\phi^0-(1-\phi)^0}{\sqrt{5}} = 0$\hfill True\\
%$n=1\rightarrow \frac{\phi^1-(1-\phi)^1}{\sqrt{5}} = \frac{\phi- 1+\phi}{\sqrt{5}}=\frac{2\phi -1}{\sqrt{5}}=\frac{1+\sqrt{5}-1}{\sqrt{5}}=1$\hfill True\\
%$Inductive~Hypothesis:~\forall k\in\mathbb{N}~(F_k = \frac{\phi^k-(1-\phi)^k}{\sqrt{5}})$\\
%$let~n=k+2$\\
%\\
%$F_{k+2} = \frac{\phi^{k+2}-(1-\phi)^{k+2}}{\sqrt{5}}
%= \frac{\phi^2\phi^{k}-(1-\phi)^{2}(1-\phi)^{k}}{\sqrt{5}}$\hfill Factor out\\
%$\phi^2 = (\frac{1+\sqrt{5}}{2})^2=\frac{6}{4}+\frac{\sqrt{5}}{2}=1 +\frac{1+\sqrt{5}}{2}
%= 1+\phi$\\
%$(1-\phi)^2=(1-\frac{1+\sqrt{5}}{2})^2=(\frac{1-\sqrt{5}}{2})^2=\frac{6}{4}-\frac{\sqrt{5}}{2}=1+\frac{1-\sqrt{5}}{2}=(1+(1-\phi))$\\
%\\
%$F_{k+2} = \frac{\phi^{k}(1+\phi)-(1-\phi)^{k}(1+(1-\phi))}{\sqrt{5}}
%= \frac{\phi^{k}+\phi^{k+1}-(1-\phi)^{k}-(1-\phi)^{k+1}}{\sqrt{5}}$\hfill Simplify\\
%$F_{k+2} =\frac{\phi^{k}-(1-\phi)^{k}}{\sqrt{5}} + 
%\frac{\phi^{k+1}-(1-\phi)^{k+1}}{\sqrt{5}}\hfill$ Group by order\\
%$F_{k+2} = F_{k} +F_{k+1}$\hfill Substitute hypothesis\\
%$F_k = F_{k-2}+F_{k-1}\hfill$ Valid
%\clearpage
%%8c
%\item
%$\forall n\geq2\in\mathbb{N}~f(n)=7f(n-1)-10f(n-2)$ where $f(0) = 1,~ f(1) = 2$\\
%$table~f(n)$\\
%$f(0) = 1$\\
%$f(1) =2$\\
%$f(2)=4$\\
%$f(3)=8$\\
%$f(4)=16$\\
%\\
%$Proof:$
%$f(n)=7f(n-1)-10f(n-2)=2^n$ \hfill Fits the table\\
%$Inductive~hypothesis:~f(n)=2^k$\\
%$let~n=k$
%$f(k)=7f(k-1)-10f(k-2)=7\times2^{k-1}-10\times2^{k-2}$\hfill Substitute hypothesis\\
%$f(k)=\frac{7\times2^{k}}{2}-\frac{10\times2^{k}}{4}
%=\frac{14\times2^{k}}{4}-\frac{10\times2^{k}}{4}
%=\frac{4\times2^{k}}{4}= 2^k$\hfill Simplify\\
%$f(k) = 2^k$\hfill Valid\\
%
%\end{enumerate}
\clearpage

\section*{Problem 9}
\begin{enumerate}[(a)]
\item
Incorrect, another base case is needed\\
If we add another base case where $n=1$ then\\
Suppose $n=1$. If max$(x,y) =1$ and $x,y\in\mathbb{N}$, then $x=1 \vee y=1$ \\
hence $(x\leq y)\vee(x>y)$\\
This shows that the claim is false due to the fact that max$(x,y)=n$ is true even when $x>y$\\
\item
Incorrect, The inductive step did not prove $n+1<2^{n+1}$.
\item
Incorrect, Proof by contrapositive would lead you to assume that $n^2 + 1$ is not a multiple of 3 which would imply that $2n+1$ is not a multiply of three. This example however did a proof of cases.\\
\end{enumerate}
\end{document}

