\documentclass[11pt,letterpaper]{article}
\usepackage{amsmath}
\usepackage{amssymb}
\usepackage{fullpage}
\usepackage{enumerate}

\title{CS 70, Summer 2014 --- Homework 3} % Put the correct homework number here.
\author{Harsimran (Sammy) Sidhu, SID 23796591} % Put your name and student ID here.

\begin{document}

\maketitle

Collaboraters: Chonyi Lama, Jenny Pushkarskaya

Sources: 

\section*{Problem 1} % Put the correct problem number here

\begin{enumerate}[(a)]
\item
$4^{273}$ mod 11 using repeated squaring\\
$4\times4^{272} \equiv 4\times16^{136} \equiv 4\times5^{136}
\equiv 4\times 25^{68} \equiv 4\times3^{68}\equiv 4\times9^{34}\equiv4\times81^{17}\equiv 4\times4^{17}$ (mod 11)\\
$4\times4^{17}\equiv 4^{18}\equiv 16^{9} \equiv 5^{9} \equiv 5\times5^{8} \equiv 5\times25^{4}\equiv 5\times3^{4} \equiv 5\times 9^{2} \equiv 5\times81 \equiv 5\times4\equiv 20$ (mod 11)\\$9$ mod 11\\

\item
$4^{273}$ mod 11, using Fermat's Little Theorem.\\
$4^{270+3}$ $\equiv$ $4^{3}\times 4^{(10\times27)}\equiv 4^{3}\times (4^{10})^{27}$ (mod 11)\\
$4^{3}\times (1)^{27}\equiv 4^{3} \equiv 64\equiv 9$ (mod 11)\\
9 mod 11\\
\item
$4^{5^{273}}$ mod 11\\
since $4^{5^{273}}$ is in mod 11, that would mean that $5^{273}$ is in mod 10 due to the fact that $5^{273}$ will only have 10 congruence classes.\\
\\
$5^{273} \equiv 5^{270+3}\equiv 5^{9\times30}\times5^{3}\equiv (5^9)^{30}\times 5^{3}\equiv 
(1)^{30}\times 5^{3}$ (mod 10)\\
$5^{3}\equiv 125\equiv 5$ (mod 10)\\
\\
So $4^{5^{273}}\equiv 4^5\equiv 4\times4^4\equiv 4\times16^2\equiv 4\times 5^2\equiv 4\times 25\equiv 4\times 3\equiv 1$ (mod 11)\\
1 (mod 11)
\item
$5^{82}$ mod 21\\
Since 21 is not a prime we have to use the size of the set of coprime integers between 1 and 20 to 21 which in this case is 12. So $5^{12}\equiv 1$ mod 21

$5^{(6\times 12 + 10)} \equiv (5^{12})^6 \times 5^{10}\equiv (1)^6 \times 25^5 \equiv 4^5\equiv 4\times4^4\equiv4\times256$ (mod 21)\\
$4\times256\equiv 4\times 4$ (mod 21)\\
16 (mod 21)
\end{enumerate}
\clearpage
\section*{Problem 2}
$p=11,~q=23,~e=7,~N=253,~(p-1)(q-1)=(11-1)(23-1)=220$\\
\begin{enumerate}[(a)]
\item
$1\equiv gcd((p-1)(q-1),e)\equiv a\times (p-1)(q-1) + d\times e$ ~~~~  mod $(p-1)(q-1)$\\
$1\equiv gcd(220,7)\equiv a\times 220 + d\times 7$ ~~~~~ mod $220$\\
\\
$gcd(220,7)\rightarrow 220 = 31\times7 + 3$\\
$gcd(7,3)\rightarrow 7= 2\times3 + 1$\\
$gcd(2,1)\rightarrow 2 = 2\times1 + 0$\\
$gcd(1,0)\rightarrow 1= 0\times0 + 1$\\
$gcd(220,7)=1$\\
\\
EGCD\\
$1 = 1\times(1) + 0\times(0)$\\
$1 = 0\times(2) + 1\times(1)$\\
$1 = 1\times(7) - 2\times(3)$\\
$1 = -2\times(220) + 63\times(7)$\\
\\
$d = 63$\\
\item
$y=x^{e}$ mod N\\
$y=44^{7}$ mod 253\\
$y=44\times44^{6} \equiv 44\times44^{6}$ (mod 253)\\
$44\times1936^{3}$ (mod 253)\\
$44\times165^{3}$ (mod 253)\\
$165\times44\times165^{2}$ (mod 253)\\
$7260\times27225$ (mod 253)\\
$176\times154$ (mod 253)\\
$y=33$ (mod 253)
\item
$x=y^{d}$ mod N\\
$x = 103^{63}$ mod 253\\
$103\times103^{62}\equiv 103\times10609^{31} \equiv 103\times236^{31}$ (mod 253)\\
$24308\times236^{30} \equiv  20\times 236^{30} \equiv 20\times55696^{15} \equiv 20\times 36^{15}$(mod 253)\\
$36\times20\times36^{14} \equiv 720\times1296^{7} \equiv 214\times 31^{7} \equiv 6634\times31^{6}$(mod 253)\\
$56\times31^{6}\equiv 56\times 961^{3}\equiv 56\times202^{3}\equiv
11312\times202^2\equiv 180\times40804\equiv 180\times71$ (mod 253)\\
$12780\equiv 130$ (mod 253)\\
$x = 130$ mod 253
\end{enumerate}
\clearpage
\section*{Problem 3}
$N=p$, and $e$ is coprime to $p-1$ 
\begin{enumerate}[(a)]
\item
If $e$ is coprime to $p-1$ that would mean that we could find the decryption key, $d$ just knowing the public key, $p$ and running the extended gcd algorithm as $gcd(p-1,e) = a\times(p-1)+d\times e$\\
\\
Once $d$ is found, we can now compute $(x^{e}$ mod p $)^{d}$ mod p. This can be computed by performing repeated squaring. This will result in $x$ mod p.\\

\item
In normal RSA encryption we use $N=pq$ where $p,q$ are two primes. So the amount of congruence classes is $(p-1)(q-1)$, and to find $d$ in RSA, we find the inverse of $e$ in mod $(p-1)(q-1)$. However in this example we are using $N=p$ which has $(p-1)$ congruence classes. So to find $d$ in this example we just find the inverse of $e$ in mod $(p-1)$. This is dangerous due to the fact that $N$ aka $p$ is public key so anyone can solve for $d$.\\
\\
Armed with this knowledge we can find $(x$ mod $p)$.\\
Using the extended gcd algorithm we can find the inverse of $e$. The function would be $d = egcd((p-1),e)$. Once $d$ is found we can use the intercepted message\\ which is $(x^e$ mod $p)$. Now we just raise the intercepted message to the $d$ power mod $p$.\\ So we have $(x^e)^d\equiv (x^{ed})= (x)$ mod $p$. We solved for $d$ as the inverse of $e$ in mod $p-1$ so we know that the exponent is just 1. Therefore the final result of the intercepted message is\\ $(x)$ mod p.
\\\\
Lemma for proof above!\\
$d$ is the inverse of $e$ in mod $(p-1)$. so $ed =1+k(p-1)$\\
$(x^{ed})= x^{1+k(p-1)} = x\times x^{k(p-1)}$ (mod $p$)\\
$x\times(1)^k \equiv x $ mod $p$ \hfill Using Fermat's Little Theorem
\item
It was proven in lecture and Note 5 that the running time for the repeated squaring algorithm is $O(n)$ operations, where n is the number of bits that $p$ is made of since it is the largest number of $p,e$. This translates to $O(\log{p})$ arithmetic operations. The extended gcd algorithm also is in $O(n)$ arithmetic operations, where n is the amount of bits in $x,y$ according to Note 5. This translates roughly to $O(\log{p})$ as well. Together we have $O(\log{p}+\log{p}) = O(\log{p})$ arithmetic operations. If we assume each arithmetic operation  is $O((\log{N})^2)$ as James did in lecture then the total runtime is to be $O((\log{p})^3)$.
\end{enumerate}
\clearpage
\section*{Problem 4}
\begin{enumerate}[(a)]
\item
The maximum degree of the product $fg$ would be the sum of the leading degrees of two polynominals. Since they are at most $d$, the sum would be $d+d$ or $2d$.\\
\item
If we have two polynominals $f,g$, we can find the product by multiplying each term of $f$ with every term of $g$ and then summing the similar terms.\\
$f(x) = a_dx^d + ... + a_1x^1 + a_0$\\
$g(x) = b_dx^d + ... + b_1x^1 + b_0$\\
$fg(x) = (a_dx^d + ... + a_1x^1 + a_0)\times( b_dx^d + ... + b_1x^1 + b_0)$\\
$fg(x) = (a_dx^d\times( b_dx^d + ... + b_1x^1 + b_0) + ... + a_1x^1\times( b_dx^d + ... + b_1x^1 + b_0) + a_0\times( b_dx^d + ... + b_1x^1 + b_0))$\\
For each term we have to do $(d+1)$ multiplications which we then do on $d+1$ terms. So we have have a total of $(d+1)^2$ multiplications.\\
After multiplying, we can analyze the degrees of each set of terms.\\
The first set of will be $(2d,2d-1,2d-2,~...~,d)$, The second would be \\
$(2d-1, 2d-2, 2d-3, ...~,d-1)$ and so on and so forth until $(d,d-1,d-2,~...~,0)$. If we look carefully we can see that $d$ appears $d$ times and $d-1$ appears $d-1$ times. This pattern goes both forwards and backwards on $d$ until $2d$ or 0 is hit respectively. so the amount of additions we have to perform is the  sum of 1 to $d$ and then back to 1.\\
\\
$(1+2+ ...+(d-1)+(d)+(d-1)+(d-2)+ ... +2+1)$
\\
num of additions = $\sum\limits_{i=1}^{d}i + \sum\limits_{i=1}^{d-1}i = \dfrac{(d)(d+1)}{2} + \dfrac{(d-1)(d)}{2} = \dfrac{(d)(d+1+d-1)}{2} = \dfrac{(d)(2d)}{2} = d^2$\\

The total amount of operations is $(d+1)^2$ multiplications and $d^2$ additions which is $O(d^2)$
\item
Since both $f,g$ are at most degree $d$ we know the product is at most degree $2d$. That would mean we would need $2d+1$ points or $t\geq 2d+1$ to perform Lagrange interpolation.\\
By the definition of multiplying a function we know that $fg(x) = f(x)\times g(x)$.\\
So to compute the point value representation of $fg$. The points would simply be\\
$(x_1,~f(x_1)\times g(x_1)),~(x_2,~f(x_2)\times g(x_2)),~...~,(x_t,~f(x_t)\times g(x_t))\hfill t\geq 2d+1$
\item
Case 1: $g(x_i) = 0$ for one the points\\
This would mean that $x_i$ is a root of $g$. If we went through and performed polynomial division, this root would still exist. Infact there would be at most $d$ roots in the denominator. This implies that there are at most $d$ holes in the function acting as vertical asymptotes. Therefore it is impossible for $f/g$ to have a real value at one of these roots.\\\\
Case 2: $g(x_i) \neq 0$ for all the points\\
In this case we can represent $f/g$ in point-value form. But the condition is that we have $t = d+1$ points where $g(x_i) \neq 0$. It would simply be\\\\
$(x_1,~f(x_1)/g(x_1)),~(x_2,~f(x_2)/g(x_2)),~...~,(x_t,~f(x_t)/g(x_t))\hfill g(x_i)\neq 0,~t= d+1$


\end{enumerate}
\clearpage
\section*{Problem 5}
\begin{enumerate}[(a)]
\item
let $d$ be the degree of some polynomial.\\
Case 1: $d < p-1$\\
Since the degree of the polynomial is less than $p-1$, the equivalent polynomial that is at most degree $p-1$ is itself.\\\\
Case 2: $d\geq p-1$\\
In this case where we have a polynomial with $d\geq p-1$ in GF($p$), we can use Fermat's Little Theorem to reduce the degree of the polynomial to at most $p-1$.\\
Looking at FLT we see that $x^{p-1} \equiv 1$ (mod $p$) $\forall x \in \{1,2,~...,~p-1\}$\\
\\
But what about the case where $x$ = 0? If $x=0$ then the polynomial terms will also become 0 giving a polynomial of degree 0 which is just a constant and also the $y$ intercept.\\
Knowing that $(d\geq p-1)$, we can rewrite $d$ as $(k(p-1)+r)$ this would give\\\\
$x^d \equiv x^{k(p-1) +r} \equiv (x^{p-1})^k \times x^{r} \equiv (1)^k \times x^r \equiv x^r$ (mod $p$)\\
Since $r$ is the remainder of $(d/(p-1))$ it has to be $(0\leq r < p-1)$. We have shown that any polynomial with a degree larger than $p-1$ can we reduced to at most degree $(p-1)$ which completes the proof.
\item
In GF($p$), we have a finite set of congruence classes which classifies inputs as well as outputs. The set for both is the same which is $\{0,~1,~2,~...,~p-1\}$. We also know that a function must have a single output for every input. This means that any $x$ can only make an appearance once in the point-value representation. So if we have a set that is size $p$ that means there is at most $p$ points. We know that Lagrange interpolation requires $d+1$ points where $d$ is the degree of the resultant polynomial. From this we can report that the polynomial in GF($p$) will be at most degree $(p-1)$.
\end{enumerate}
\clearpage
\section*{Problem 6}

Let's call the secret we want to share $S$, we want to divide it up so that we need two groups out of 3 in which for each you need 4 out of 7 people in a group. Let's divide the groups into group $A,~B,~C$. each group has 7 people who we call $(A1,~A2,~ ...,~ A7),~(B1,~B2,~ ...,~ B7),~(C1,~C2,~ ...,~ C7)$.\\
Since we only need 2 groups to unlock the secret we can use a degree 1 polynomial which is a line to share $S$ to all 3 groups.\\ The function comes out to be $y=mx + S$ for some $m$. \\We now get 3 points from this function. $(a,~S+am),~(b,~S+bm),~(c,~S+cm).$ $(a\neq b\neq c)$. We now share each  $x$ coordinate with each group respectively and make each $y$ a secret to be shared among each 7 per group.
Let's call each $y$ coordinate $(S_A,~S_B,~S_C)$ respectively. We need 4 people in a group to solve one of these sub-secrets. That would mean we need to use 3 unique polynomials each of degree 3.\\
\\
$y_A = a_3x^3 + a_2x^2 + a_1x + S_A$ For some $(a_3,~a_2,~a_1)$\\
$y_B = b_3x^3 + b_2x^2 + b_1x + S_B$ For some $(b_3,~b_2,~b_1)$\\
$y_C = c_3x^3 + c_2x^2 + c_1x + S_C$ For some $(c_3,~c_2,~c_1)$\\
\\
Now we give out a unique point to each member of each group using their group function.\\
If the members wanted to figure out the secret they would have to form 2 groups of 4 and each 4 would interpolate their points to solve for their group function. Once they did that, they would use the $y$ intercept as well as the $x$ that was published to their group to interpolate with another group who also has a point to solve for $S$.
\clearpage
\section*{Problem 7}
We have a polynomial that is at most a degree of 2. \\We also know that $P(0)=4,~P(3)=1,~P(4)=2.$\\
$(0,4),~(3,1),~(4,2)$
\\\\
$\Delta_1 = \dfrac{(x-3)(x-4)}{(0-3)(0-4)} = \dfrac{(x-3)(x-4)}{12} = 12^{-1}(x-3)(x-4)$   (mod 5)\\\\
$\Delta_2 = \dfrac{(x-0)(x-4)}{(3-0)(3-4)} = \dfrac{(x)(x-4)}{-3}= (-3)^{-1}(x)(x-4)$ (mod 5)\\\\
$\Delta_3 = \dfrac{(x-0)(x-3)}{(4-0)(4-3)} = \dfrac{(x)(x-3)}{4} = 4^{-1}(x)(x-3)$ (mod 5)\\\\
Finding an alternate inverse of 12.
If we multiply 12 by 3... $(12\times3=36)$ (mod 5) we get 1. So 3 is also an inverse of 12 mod 5.\\\\
If we multiply -3 by -2... $(-3\times-2=6)$ (mod 5) we get 1. So $-2$ is also an inverse of -3 mod 5.\\\\
If we multiply 4 by 4... $(4\times4=16)$ (mod 5) we get 1. So $4$ is also an inverse of 4 mod 5.\\\\
Now we can replace the inverses with the newly calculated ones.\\\\
$\Delta_1 = 3(x-3)(x-4)$   (mod 5)\\\\
$\Delta_2 =-2(x)(x-4)$ (mod 5)\\\\
$\Delta_3 = 4(x)(x-3)$ (mod 5)\\\\
$y = y_1\Delta_1 + y_2\Delta_2+ y_2\Delta_2 = 4\times3(x-3)(x-4) + 1\times -2(x)(x-4) + 2\times 4(x)(x-3)$ mod 5\\
$y=12(x-3)(x-4)-2(x)(x-4)+8(x)(x-3)$ mod 5\\
$y= 12(x^2-7x+12)-2(x^2-4x)+8(x^2-3x)$ mod 5\\
$y= (12x^2-84x+144)+(-2x^2+8x)+(8x^2-24x)$ mod 5\\
$y= (18x^2-100x+144)$ mod 5\\
$y = 3x^2 + 4$ mod 5\\
$a_0 = P(0)= 3(0)^2 + 4 = 4$ mod 5\\
$a_1 = P(1)= 3(1)^2 + 4= 7=2$ mod 5\\
$a_2 = P(2)= 3(2)^2 + 4 = 6=1$ mod 5\\
\\
$(a_0,~a_1,~a_2) = (4,~2,~1)$
\end{document}

